\section{文字列}

\subsection{KMP}

作成されるテーブルは「パターン中の先頭$n$文字を考えたときに、そのprefix
とsuffixが一致するような部分長さ $L ~ (0 \leq L < n)$ のうち最長のもの」
を保持する。
$\mathrm{O}(n + m)$.

\begin{lstlisting}
vector<int> kmp(string s) {
  int n = s.size();
  vector<int> table(n+1);
  int q = table[0] = -1;
  for(int p = 1; p <= n; p++) {
    while(q >= 0 && s[q] != s[p-1])
      q = table[q];
    table[p] = ++q;
  }
  return table;
}
\end{lstlisting}


\subsection{Aho-Corasick}

\begin{lstlisting}
struct AC {
  AC* fail;
  AC* next[4];
  vector<int> accepts;
 
  AC() : fail(0) {
    memset(next, 0, sizeof(next));
  }
  void insert(const int* p, int id) {
    if (*p < 0)
      accepts.push_back(id);
    else
      (next[*p] ?: (next[*p] = new AC()))->insert(p+1, id);
  }
  void make() {
    this->fail = 0; // this is root
    queue<AC*> q;
    q.push(this);
    while(!q.empty()) {
      AC* cur = q.front();
      q.pop();
      REP(k, 4) {
        AC*& next = cur->next[k];
        if (!next) {
          next = (cur->fail ? cur->fail->next[k] : this);
        }
        else {
          AC* fail = cur->fail;
          while(fail && !fail->next[k])
            fail = fail->fail;
          fail = (fail ? fail->next[k] : this);
          next->fail = fail;
          next->accepts.insert(next->accepts.end(), ALLOF(fail->accepts));
          q.push(next);
        }
      }
    }
  }
};
\end{lstlisting}

\subsection{Suffix Array}

\subsubsection{Larsson Sadakane}

たぶん $\mathrm{O}(n \log n )$.

\begin{lstlisting}
int* suffix_array(unsigned char* str) {
  int n = strlen((char*)str);
  int *v, *u, *g, *b, r[256];
  v = new int[n+1]; u = new int[n+1]; g = new int[n+1]; b = new int[n+1];
  for(int i = 0; i <= n; i++) {
    v[i] = i;
    g[i] = str[i];
  }
  for(int h = 1; ; h *= 2) {
    int m = *max_element(g, g+n+1);
    for(int k = h; k >= 0; k -= h) {
      for(int ord = 0; m >> ord; ord += 8) {
        memset(r, 0, sizeof(r));
        for(int i = 0; i <= n; i++)
          r[(g[min(v[i]+k, n)] >> ord) & 0xff]++;
        for(int i = 1; i < 256; i++)
          r[i] += r[i-1];
        for(int i = n; i >= 0; i--)
          u[ --r[(g[min(v[i]+k, n)] >> ord) & 0xff] ] = v[i];
        swap(u, v);
      }
    }
    b[0] = 0;
    for(int i = 1; i <= n; i++)
      b[i] = b[i-1] + ((g[v[i-1]] != g[v[i]] ? g[v[i-1]] < g[v[i]] : g[v[i-1]+h] < g[v[i]+h]) ? 1 : 0);
    if (b[n] == n)
      break;
    for(int i = 0; i <= n; i++)
      g[v[i]] = b[i];
  }
  delete[] g; delete[] b; delete[] u;
  return v;
}
\end{lstlisting}


\subsubsection{Ka\"rkka\"inen Sanders}

テキストサイズ $\geq$ 2.

\begin{lstlisting}
// Derived from:
// J. Karkkainen and P. Sanders: Simple Linear Work Suffix Array Construction.
// In 30th International Colloquium on Automata, Languages and Programming,
// number 2719 in LNCS, pp. 943--955, Springer, 2003.
 
void radix_sort(int* src, int* dest, int* rank, int n, int m) {
  int* cnt = new int[m+1];
  memset(cnt, 0, sizeof(int)*(m+1));
  for(int i = 0; i < n; i++)
    cnt[rank[src[i]]]++;
  for(int i = 1; i <= m; i++)
    cnt[i] += cnt[i-1];
  for(int i = n-1; i >= 0; i--)
    dest[--cnt[rank[src[i]]]] = src[i];
}
 
bool triple_less(int a0, int a1, int a2, int b0, int b1, int b2) {
  if (a0 != b0)
    return (a0 < b0);
  if (a1 != b1)
    return (a1 < b1);
  return (a2 < b2);
}
 
void suffix_array(int* text, int* sa, int n, int m) {
  assert(n >= 2);
 
  int n0 = (n+2)/3, n1 = (n+1)/3, n2 = n/3, n02 = n0 + n2;
 
  int* rank12 = new int[n02+3]; rank12[n02+0] = rank12[n02+1] = rank12[n02+2] = 0;
  int* sa12 = new int[n02+3]; sa12[n02+0] = sa12[n02+1] = sa12[n02+2] = 0;
  int* tmp0 = new int[n0];
  int* sa0 = new int[n0];
 
  for(int i = 0; i < n02; i++)
    rank12[i] = i*3/2+1;
  radix_sort(rank12, sa12, text+2, n02, m);
  radix_sort(sa12, rank12, text+1, n02, m);
  radix_sort(rank12, sa12, text+0, n02, m);
 
  int name = 0;
  for(int i = 0; i < n02; i++) {
    if (name == 0 ||
        text[sa12[i]+0] != text[sa12[i-1]+0] ||
        text[sa12[i]+1] != text[sa12[i-1]+1] ||
        text[sa12[i]+2] != text[sa12[i-1]+2])
      name++;
    rank12[sa12[i]/3 + (sa12[i]%3 == 1 ? 0 : n0)] = name;
  }
 
  if (name < n02) {
    suffix_array(rank12, sa12, n02, name);
    for(int i = 0; i < n02; i++)
      rank12[sa12[i]] = i+1;
  }
  else {
    for(int i = 0; i < n02; i++)
      sa12[rank12[i]-1] = i;
  }
 
  for(int i = 0, j = 0; i < n02; i++)
    if (sa12[i] < n0)
      tmp0[j++] = 3*sa12[i];
  radix_sort(tmp0, sa0, text, n0, m);
 
  for(int i = 0, j = n0-n1, k = 0; k < n; ) {
    int a = sa0[i];
    int b = (sa12[j] < n0 ? sa12[j]*3+1 : (sa12[j]-n0)*3+2);
    if (b%3 == 1 ?
          triple_less(text[a], rank12[a/3], 0,
                      text[b], rank12[b/3+n0], 0) :
                      triple_less(text[a], text[a+1], rank12[a/3+n0],
                                  text[b], text[b+1], rank12[b/3+1])) {
      sa[k++] = a;
      if (++i == n0)
        while(k < n)
          sa[k++] = (sa12[j] < n0 ? sa12[j++]*3+1 : (sa12[j++]-n0)*3+2);
    }
    else {
      sa[k++] = b;
      if (++j == n02)
        while(k < n)
          sa[k++] = sa0[i++];
    }
  }
 
  delete[] rank12; delete[] sa12; delete[] tmp0; delete[] sa0;
}
\end{lstlisting}


\subsubsection{高さ配列}

Kasaiらによる、構築されたSuffix Arrayについて、順位が隣り合った二つのsuffixのlongest common prefixを求めるアルゴリズム。

\begin{lstlisting}
int* height_array(char* text, int* sa, int n) {
  int* height = new int[n];
  int* rank = new int[n];
  REP(i, n)
    rank[sa[i]] = i;
  int h = 0;
  height[0] = 0;
  REP(i, n) {
    if (rank[i] > 0) {
      int k = sa[rank[i]-1];
      while(text[i+h] == text[k+h])
        h++;
      height[rank[i]] = h;
      if (h > 0)
        h--;
    }
  }
  delete[] rank;
  return height;
}
\end{lstlisting}



\newpage
