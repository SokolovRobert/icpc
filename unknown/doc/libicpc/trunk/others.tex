\section{その他}

\subsection{ビット演算}

\begin{lstlisting}
unsigned int __builtin_popcount(unsigned int);  // 1であるビットの数を数える。
unsigned long long __builtin_popcountll(unsigned long long);
unsigned int __builtin_ctz(unsigned int);    // 末尾の0であるビットの数を数える。
unsigned long long __builtin_ctzll(unsigned long long);
y = x&-x;   // 最右ビットを抜き出す。
y = x&(x-1); // 最右ビットを落とす。
#define FOR_SUBSET(b, a) for(int b = (a)&-(a); b != 0; b = (((b|~(a))+1)&(a)))
#define FOR_SUBSET(b, a) for(int b = a; b != 0; b = (b-1)&a)
int next_combination(int p) {
  int lsb = p&-p;
  int rem = p+lsb;
  int rit = rem&~p;
  return rem|(((rit/lsb)>>1)-1);
}
\end{lstlisting}


\subsection{平衡木}

Treap. \w{サブツリーに対する操作を行ってはいけない。}

\begin{lstlisting}
template<class Key, class Value, class Cache>
struct Treap {
  Key key;
  Value value;
  int prio;
  Treap* ch[2];
  bool cached;
  Cache cache;
  Treap(const Key& key, const Value& value) : key(key), value(value), prio(rand()), cached(false) {
    ch[0] = ch[1] = 0; // is this necessary?
  }
  Treap* insert(const Key& newkey, const Value& newvalue) {
    if (!this)
      return new Treap(newkey, newvalue);
    if (newkey == key)
      return this;
    int side = (newkey < key ? 0 : 1);
    ch[side] = ch[side]->insert(newkey, newvalue);
    cached = false;
    return rotate(side);
  }
  Treap* rotate(int side) {
    if (ch[side] && ch[side]->prio > prio) {
      Treap* rot = ch[side];
      this->ch[side] = rot->ch[side^1];
      rot->ch[side^1] = this;
      this->cached = rot->cached = false;
      return rot;
    }
    return this;
  }
  void clear() {
    if (this) {
      ch[0]->clear();
      ch[1]->clear();
      delete this;
    }
  }
  Treap* remove(const Key& oldkey) {
    if (!this)
      return this;
    if (key == oldkey) {
      if (!ch[1]) {
        Treap* res = ch[0];
        delete this;
        return res;
      }
      pair<Treap*, Treap*> res = ch[1]->delete_min();
      res.first->ch[0] = this->ch[0];
      res.first->ch[1] = res.second;
      res.first->cached = false;
      delete this;
      return res.first->balance();
    }
    int side = (oldkey < key ? 0 : 1);
    ch[side] = ch[side]->remove(oldkey);
    cached = false;
    return this;
  }
  pair<Treap*, Treap*> delete_min() {
    cached = false;
    if (!ch[0])
      return make_pair(this, ch[1]);
    pair<Treap*, Treap*> res = ch[0]->delete_min();
    ch[0] = res.second;
    return make_pair(res.first, this);
  }
  inline int priority() {
    return (this ? prio : -1);
  }
  Treap* balance() {
    int side = (ch[0]->priority() > ch[1]->priority() ? 0 : 1);
    Treap* rot = rotate(side);
    if (rot == this)
      return this;
    return rot->balance();
  }
  Cache eval() {
    if (!this)
      return Cache();
    if (!cached)
      cache = Cache(key, value, ch[0]->eval(), ch[1]->eval());
    cached = true;
    return cache;
  }
};

// 小さいほうからn番目の要素をクエリできるようにする場合の例
struct SizeCache {
  int size;
  SizeCache() : size(0) {}
  template<class Key, class Value>
  SizeCache(const Key& key, const Value& value, const SizeCache& left, const SizeCache& right)
   : size(left.size+right.size+1) {}
};
template<class Key, class Value>
pair<Key, Value> nth(Treap<Key, Value, SizeCache>* root, int k) {
  int l = root->ch[0]->eval().size;
  if (k < l)
    return nth<Key, Value>(root->ch[0], k);
  if (k == l)
    return make_pair(root->key, root->value);
  return nth<Key, Value>(root->ch[1], k-(l+1));
}
\end{lstlisting}


\subsection{Fenwick Tree}

配列のpartial sumの取得と要素の書き換えをそれぞれ対数時間で行うデータ構造。別名Binary Indexed Tree。
n次元にも容易に拡張できる。
和の取得・要素の更新ともに$\mathrm{O}(\log n)$.

\begin{lstlisting}
T bitquery(const vector<T>& bit, int from, int to) { // [from, to)
  if (from > 0)
    return bitquery(bit, 0, to) - bitquery(bit, 0, from);
  T res = T();
  for(int k = to-1; k >= 0; k = (k & (k+1)) - 1)
    res += bit[k];
  return res;
}

void bitupdate(vector<T>& bit, int pos, const T& delta) {
  for(const int n = bit.size(); pos < n; pos |= pos+1)
    bit[pos] += delta;
}
\end{lstlisting}

\subsection{Range Minimum Query}

列のある区間の最小値を対数時間で求めるデータ構造。
\w{配列サイズは2のべき乗であること。}
初期化に$\mathrm{O}(n)$. 更新・クエリに$\mathrm{O}(\log n)$.

\begin{lstlisting}
// 配列を拡張してRMQに対応させる
void rmq_ext(vector<T>& v) {
  int n = v.size();
  assert(__builtin_popcount(n) == 1); // 長さは2のべき乗でなければならない
  v.resize(n*2);
  for(int i = n; i < 2*n; i++)
    v[i] = min(v[(i-n)*2+0], v[(i-n)*2+1]);
}
// 列の要素を書き換える
void rmq_update(vector<T>& rmq, int pos, const T& value) {
  int n = rmq.size() / 2;
  rmq[pos] = value;
  while(pos < 2*n-1) {
    rmq[pos/2+n] = min(rmq[pos], rmq[pos^1]);
    pos = pos/2+n;
  }
}
// [from, to)の最小値を取り出す
T rmq_query(const vector<T>& rmq, int from, int to) {
  int n = rmq.size() / 2;
  int p = min((from == 0 ? 32 : __builtin_ctz(from)), 31-__builtin_clz(to-from));
  T x = rmq[(from>>p)|((n*2*((1<<p)-1))>>p)];
  from += 1<<p;
  if (from < to)
    x = min(x, rmq_query(rmq, from, to));
  return x;
}
\end{lstlisting}


\subsection{Range Tree}

区間に対する加算、区間中の最小値のクエリ、各要素の値のクエリを対数程度の時間で行う木。
\verb|range_add| に $O((\log n)^2)$.\verb|range_min| に $O(\log n)$.
\verb|query| に $O(\log n)$.

\begin{lstlisting}
struct range_tree {
  int m;
  vector< pair<int, int> > tree;

  range_tree(int n) {
    m = n;
    while(m&(m-1))
      m += m&-m;
    tree.assign(2*m, make_pair(0, 0));
    for(int k = n; k < m; k++)
      tree[k] = make_pair(INF, INF);
    for(int k = m; k < 2*m-1; k++)
      tree[k] = make_pair(0,
                min(tree[((k&~m)<<1)^0].second,
                  tree[((k&~m)<<1)^1].second));
  }
  void range_add(int a, int b, int d) {
    while(a < b) {
      int r = 1, k = a;
      while((a & r) == 0 && a + (r<<1) <= b) {
        r <<= 1;
        k = (k >> 1) | m;
      }
      tree[k].first += d;
      do {
        tree[k].second = tree[k].first +
          (k < m ? zero :
           min(tree[((k&~m)<<1)^0].second,
             tree[((k&~m)<<1)^1].second));
        k = (k >> 1) | m;
      } while(k < 2*m-1);
      a += r;
    }
  }
  int range_min(int a, int b) {
    int res = INF;
    while(a < b) {
      int r = 1, k = a;
      while((a & r) == 0 && a + (r<<1) <= b) {
        r <<= 1;
        k = (k >> 1) | m;
      }
      res = min(res, tree[k].second);
      a += r;
    }
    return res;
  }
  int query(int k) {
    int res = 0;
    while(k < 2*m-1) {
      res += tree[k].first;
      k = (k >> 1) | m;
    }
    return res;
  }
};
\end{lstlisting}



\subsection{α-β枝狩り}

なんだかんだで毎回書くのに苦労するあれ。

\begin{lstlisting}
// [alpha..beta]の値以外を返しても意味を持たない、という意味
int search(int alpha = -INF, int beta = INF) {
  if (finished())
    return 0;
  if (enemy_turn())
    return -search(-beta, -alpha);
  for(move a : possible moves) {
    int p = gain(a);
    alpha >?= p + search(alpha-p, beta-p);
    if (alpha >= beta)
      break;
  }
  return alpha;
}
\end{lstlisting}



\subsection{マトロイド交差}

マトロイド交差の最大独立集合を求める。
\w{M1, M2について独立性判定ができること(M::appendable)。
M2について、独立な集合に要素を加えて非独立になったとき、存在する一意の回路を計算できること(M2::circuit)。}

\begin{lstlisting}
template<class E, class M1, class M2>
set<E> augment(M1 m1, M2 m2, set<E> g, set<E> s) {
  map<E, E> trace;

  vector<E> rights;
  FOR(i, g) {
    if (m1.appendable(s, *i)) {
      rights.push_back(*i);
      trace[*i] = *i;
    }
  }
  while(!rights.empty()) {
    vector<E> lefts;
    REP(i, rights.size()) {
      E e = rights[i];
      if (m2.appendable(s, e)) {
        while(trace[e] != e) {
          s.insert(e); e = trace[e];
          s.erase(e); e = trace[e];
        }
        s.insert(e);
        return s;
      }
      vector<E> c = m2.circuit(s, e);
      REP(j, c.size()) {
        E f = c[j];
        if (trace.count(f) == 0) {
          trace[f] = e;
          lefts.push_back(f);
        }
      }
    }
    rights.clear();
    REP(i, lefts.size()) {
      E f = lefts[i];
      s.erase(f);
      FOR(i, g) {
        E e = *i;
        if (m1.appendable(s, e) && trace.count(e) == 0) {
          trace[e] = f;
          rights.push_back(e);
        }
      }
      s.insert(f);
    }
  }
  return s;
}
\end{lstlisting}



\subsection{Zellerの公式}

\begin{lstlisting}
int zeller(int y, int m, int d) {
  if (m <= 2) {
    y--;
    m += 12;
  }
  return (y + y/4 - y/100 + y/400 + (13 * m + 8) / 5 + d) % 7;
}
\end{lstlisting}


\subsection{Grundy数}

moveできなくなった時に負けるゲームについて、
勝てるかどうかの判定と必勝法の計算を行う道具。
盤面ひとつひとつに数字を割り当てる。後手必勝が0、先手必勝が1以上。
選択はmex。平行していくつかのゲームを進める場合はxor。


\newpage


